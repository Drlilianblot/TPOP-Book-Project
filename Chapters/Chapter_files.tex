\chapterimage{chapter_head_1.pdf} % Chapter heading image
\chapter{Files}

\index{file}
\index{type!file}


\section{Persistence}

\index{persistence}

Most of the programs we have seen so far are transient in the
sense that they run for a short time and produce some output,
but when they end, their data disappears.  If you run the program
again, it starts with a clean slate.

Other programs are {\bf persistent}: they run for a long time
(or all the time); they keep at least some of their data
in permanent storage (a hard drive, for example); and
if they shut down and restart, they pick up where they left off.
%
Examples of persistent programs are operating systems, which
run pretty much whenever a computer is on, and web servers,
which run all the time, waiting for requests to come in on
the network.
%
One of the simplest ways for programs to maintain their data
is by reading and writing text files. A text file is a sequence
of characters stored on a permanent medium like a hard drive, 
flash memory, or CD-ROM. 


\section{Reading from a file}
\label{wordlist}

In this chapter we will be using a list of English words.
There are lots of word lists available on the Web, but the one most
suitable for our purpose is one of the word lists collected and
contributed to the public domain by Grady Ward as part of the Moby
lexicon project\footnote{\url{wikipedia.org/wiki/Moby_Project}.}.  It
is a list of 113,809 official crosswords; that is, words that are
considered valid in crossword puzzles and other word games.  In the
Moby collection, the filename is {\tt 113809of.fic}; I include a copy
of this file, with the simpler name {\tt words.txt}, along with
Swampy.

\index{Swampy}
\index{crosswords}

This file is in plain text, so you can open it with a text
editor, but you can also read it from Python.  The built-in
function {\tt open} takes the name of the file as a parameter
and returns a {\bf file object} you can use to read the file.

\index{open function}
\index{function!open}
\index{plain text}
\index{text!plain}
\index{object!file}
\index{file object}

\beforeverb
\begin{pyinterpreter}
>>> fin = open('words.txt')
>>> print(fin)
<open file 'words.txt', mode 'r' at 0xb7f4b380>
\end{pyinterpreter}
\afterverb
%
{\tt fin} is a common name for a file object used for
input.  Mode \verb"'r'" indicates that this file is open for
reading (as opposed to \verb"'w'" for writing).

\index{readline method}
\index{method!readline}

The file object provides several methods for reading, including
{\tt readline}, which reads characters from the file
until it gets to a newline and returns the result as a
string:

\beforeverb
\begin{pyinterpreter}
>>> fin.readline()
'aa\r\n'
\end{pyinterpreter}
\afterverb
%
The first word in this particular list is ``aa,'' which is a kind of
lava.  The sequence \verb"\r\n" represents two whitespace characters,
a carriage return and a newline, that separate this word from the
next.

The file object keeps track of where it is in the file, so
if you call {\tt readline} again, you get the next word:

\beforeverb
\begin{pyinterpreter}
>>> fin.readline()
'aah\r\n'
\end{pyinterpreter}
\afterverb
%
The next word is ``aah,'' which is, believe it or not, a perfectly legitimate
word. If the whitespace is bothering you,
we can get rid of it with the string method {\tt strip}:

\index{strip method}
\index{method!strip}

\beforeverb
\begin{pyinterpreter}
>>> line = fin.readline()
>>> word = line.strip()
>>> print(word)
aahed
\end{pyinterpreter}
\afterverb
%
You can also use a file object as part of a {\tt for} loop.
This program reads {\tt words.txt} and prints each word, one
per line:

\index{open function}
\index{function!open}

\beforeverb
\begin{pycode}
fin = open('words.txt')
for line in fin:
    word = line.strip()
    print(word)
\end{pycode}
\afterverb
%

\section{Writing to a file}

\index{file!writing}

\index{open function}
\index{function!open}

To write a file, you have to open it with mode
\verb"'w'" as a second parameter:

\beforeverb
\begin{pyinterpreter}
>>> fout = open('output.txt', 'w')
>>> print(fout)
<open file 'output.txt', mode 'w' at 0xb7eb2410>
\end{pyinterpreter}
\afterverb
%
If the file already exists, opening it in write mode clears out
the old data and starts fresh, so be careful!
If the file doesn't exist, a new one is created.

The {\tt write} method puts data into the file.

\beforeverb
\begin{pyinterpreter}
>>> line1 = "This here's the wattle,\n"
>>> fout.write(line1)
\end{pyinterpreter}
\afterverb
%
Again, the file object keeps track of where it is, so if
you call {\tt write} again, it adds the new data to the end.

\beforeverb
\begin{pyinterpreter}
>>> line2 = "the emblem of our land.\n"
>>> fout.write(line2)
\end{pyinterpreter}
\afterverb
%
When you are done writing, you have to close the file.

\beforeverb
\begin{pyinterpreter}
>>> fout.close()
\end{pyinterpreter}
\afterverb
%

\index{close method}
\index{method!close}


\section{Format operator}

\index{format operator}
\index{operator!format}

The argument of {\tt write} has to be a string, so if we want
to put other values in a file, we have to convert them to
strings.  The easiest way to do that is with {\tt str}:

\beforeverb
\begin{pyinterpreter}
>>> x = 52
>>> f.write(str(x))
\end{pyinterpreter}
\afterverb
%
An alternative is to use the {\bf format operator}, {\tt \%}.  When
applied to integers, {\tt \%} is the modulus operator.  But
when the first operand is a string, {\tt \%} is the format operator.

\index{format string}

The first operand is the {\bf format string}, which contains
one or more {\bf format sequences}, which
specify how
the second operand is formatted.  The result is a string.

\index{format sequence}

For example, the format sequence \verb"'%d'" means that
the second operand should be formatted as an
integer ({\tt d} stands for ``decimal''):

\beforeverb
\begin{pyinterpreter}
>>> camels = 42
>>> '%d' % camels
'42'
\end{pyinterpreter}
\afterverb
%
The result is the string \verb"'42'", which is not to be confused
with the integer value {\tt 42}.

A format sequence can appear anywhere in the string,
so you can embed a value in a sentence:

\beforeverb
\begin{pyinterpreter}
>>> camels = 42
>>> 'I have spotted %d camels.' % camels
'I have spotted 42 camels.'
\end{pyinterpreter}
\afterverb
%
If there is more than one format sequence in the string,
the second argument has to be a tuple.  Each format sequence is
matched with an element of the tuple, in order.

The following example uses \verb"'%d'" to format an integer,
\verb"'%g'" to format
a floating-point number (don't ask why), and \verb"'%s'" to format
a string:

\beforeverb
\begin{pyinterpreter}
>>> 'In %d years I have spotted %g %s.' % (3, 0.1, 'camels')
'In 3 years I have spotted 0.1 camels.'
\end{pyinterpreter}
\afterverb
%
The number of elements in the tuple has to match the number
of format sequences in the string.  Also, the types of the
elements have to match the format sequences:

\index{exception!TypeError}
\index{TypeError}

\beforeverb
\begin{pyinterpreter}
>>> '%d %d %d' % (1, 2)
TypeError: not enough arguments for format string
>>> '%d' % 'dollars'
TypeError: illegal argument type for built-in operation
\end{pyinterpreter}
\afterverb
%
In the first example, there aren't enough elements; in the
second, the element is the wrong type.

The format operator is powerful, but it can be difficult to use.  You
can read more about it at
\url{docs.python.org/lib/typesseq-strings.html}.

% You can specify the number of digits as part of the format sequence.
% For example, the sequence \verb"'%8.2f'"
% formats a floating-point number to be 8 characters long, with
% 2 digits after the decimal point:

% \beforeverb
% \begin{pyinterpreter}
% >>> '%8.2f' % 3.14159
% '    3.14'
% \end{pyinterpreter}
% \afterverb
% %
% The result takes up eight spaces with two
% digits after the decimal point.  


\section{Filenames and paths}
\label{paths}

\index{filename}
\index{path}
\index{directory}
\index{folder}

Files are organized into {\bf directories} (also called ``folders'').
Every running program has a ``current directory,'' which is the
default directory for most operations.  
For example, when you open a file for reading, Python looks for it in the
current directory.

\index{os module}
\index{module!os}

The {\tt os} module provides functions for working with files and
directories (``os'' stands for ``operating system'').  {\tt os.getcwd}
returns the name of the current directory:

\index{getcwd function}
\index{function!getcwd}

\beforeverb
\begin{pyinterpreter}
>>> import os
>>> cwd = os.getcwd()
>>> print(cwd)
/home/dinsdale
\end{pyinterpreter}
\afterverb
%
{\tt cwd} stands for ``current working directory.''  The result in
this example is {\tt /home/dinsdale}, which is the home directory of a
user named {\tt dinsdale}.

\index{working directory}
\index{directory!working}

A string like {\tt cwd} that identifies a file is called a {\bf path}.
A {\bf relative path} starts from the current directory;
an {\bf absolute path} starts from the topmost directory in the
file system.

\index{relative path}
\index{path!relative}
\index{absolute path}
\index{path!absolute}

The paths we have seen so far are simple filenames, so they are
relative to the current directory.  To find the absolute path to
a file, you can use {\tt os.path.abspath}:

\beforeverb
\begin{pyinterpreter}
>>> os.path.abspath('memo.txt')
'/home/dinsdale/memo.txt'
\end{pyinterpreter}
\afterverb
%
{\tt os.path.exists} checks
whether a file or directory exists:

\index{exists function}
\index{function!exists}

\beforeverb
\begin{pyinterpreter}
>>> os.path.exists('memo.txt')
True
\end{pyinterpreter}
\afterverb
%
If it exists, {\tt os.path.isdir} checks whether it's a directory:

\beforeverb
\begin{pyinterpreter}
>>> os.path.isdir('memo.txt')
False
>>> os.path.isdir('music')
True
\end{pyinterpreter}
\afterverb
%
Similarly, {\tt os.path.isfile} checks whether it's a file.

{\tt os.listdir} returns a list of the files (and other directories)
in the given directory:

\beforeverb
\begin{pyinterpreter}
>>> os.listdir(cwd)
['music', 'photos', 'memo.txt']
\end{pyinterpreter}
\afterverb
%
To demonstrate these functions, the following example
``walks'' through a directory, prints
the names of all the files, and calls itself recursively on
all the directories.

\index{walk, directory}
\index{directory!walk}

\beforeverb
\begin{pycode}
def walk(dir):
    for name in os.listdir(dir):
        path = os.path.join(dir, name)

        if os.path.isfile(path):
            print(path)
        else:
            walk(path)
\end{pycode}
\afterverb
%
{\tt os.path.join} takes a directory and a file name and joins
them into a complete path.  

\begin{exercise}
Modify {\tt walk} so that instead of printing the names of
the files, it returns a list of names.
\end{exercise}

\begin{exercise}
The {\tt os} module provides a function called {\tt walk}
that is similar to this one but more versatile.  Read
the documentation and use it to print the names of the
files in a given directory and its subdirectories.
\end{exercise}


\section{Catching exceptions}
\label{catch}

A lot of things can go wrong when you try to read and write
files.  If you try to open a file that doesn't exist, you get an
{\tt IOError}:

\index{open function}
\index{function!open}
\index{exception!IOError}
\index{IOError}

\beforeverb
\begin{pyinterpreter}
>>> fin = open('bad_file')
IOError: [Errno 2] No such file or directory: 'bad_file'
\end{pyinterpreter}
\afterverb
%
If you don't have permission to access a file:

\index{file!permission}
\index{permission, file}

\beforeverb
\begin{pyinterpreter}
>>> fout = open('/etc/passwd', 'w')
IOError: [Errno 13] Permission denied: '/etc/passwd'
\end{pyinterpreter}
\afterverb
%
And if you try to open a directory for reading, you get

\beforeverb
\begin{pyinterpreter}
>>> fin = open('/home')
IOError: [Errno 21] Is a directory
\end{pyinterpreter}
\afterverb
%
To avoid these errors, you could use functions like {\tt os.path.exists}
and {\tt os.path.isfile}, but it would take a lot of time and code
to check all the possibilities (if ``{\tt Errno 21}'' is any
indication, there are at least 21 things that can go wrong).

\index{exception, catching}
\index{try statement}
\index{statement!try}

It is better to go ahead and try, and deal with problems if they
happen, which is exactly what the {\tt try} statement does.  The
syntax is similar to an {\tt if} statement:

\beforeverb
\begin{pycode}
try:    
    fin = open('bad_file')
    for line in fin:
        print(line)
    fin.close()
except:
    print('Something went wrong.')
\end{pycode}
\afterverb
%
Python starts by executing the {\tt try} clause.  If all goes
well, it skips the {\tt except} clause and proceeds.  If an
exception occurs, it jumps out of the {\tt try} clause and
executes the {\tt except} clause.

Handling an exception with a {\tt try} statement is called {\bf
catching} an exception.  In this example, the {\tt except} clause
prints an error message that is not very helpful.  In general,
catching an exception gives you a chance to fix the problem, or try
again, or at least end the program gracefully.


\subsection{{\tt finally} clause}
There are two options to the \verb|try-except| statement.
The first one is the clause {\tt finally}. All code included in the 
{\tt finally} clause will be executed, whether an exception occurred 
or not. This is a good place to clean up our program. For example
this is a good place to close a file if it has been open in the {\tt try} 
clause. The code below shows how it is done.


\beforeverb
\begin{pycode}
fin = None
try:    
    print('Try to open a file.')
    fin = open('words.txt')
    for line in fin:
        print(line)
except:
    print('Something went wrong.')
finally:
    print('cleaning up')
    if fin is not None:
        print('closing the file')
        fin.close()
\end{pycode}
\afterverb

First we set the variable {\tt fin} to {\tt None}. In the {\tt try} clause, we 
open a file and then assign it to {\tt fin}. Two things can happen:

\begin{itemize}
	\item An exception occurs while we
are trying to open the file, {\tt fin} is not assigned any new object and contains 
the value {\tt None}. The program jumps directly to the {\tt except} clause and
executes the code in the {\tt except} block. Then the program jumps to the {\tt finally} 
clause and executes the code there.

	\item The file is opened successfully, {\tt fin} is assigned the file object, and the rest
	of the code in the {\tt try} statement is executed. Then the program jumps to the {\tt finally}
	clause and executes the code there.

\end{itemize} 

\subsection{{\tt else} clause}
The second option is the {\tt else} clause, which should be after the {\tt except} clause and before the {\tt finally} clause. The code in the {\tt else} clause is executed only if no exceptions were raised. It is executed before the {\tt finally} clause. The {\tt else} clause is use for all code that does not raise any exception. In our previous code, it would be the place to read the lines in the file. The refactored code is:

\beforeverb
\begin{pycode}
fin = None
try:    
    fin = open('word.txt')
except:
    print('Something went wrong.')
else:
    print('Do your thing if all went well.')
    for line in fin:
        print(line)    
finally:
    print('cleaning up.')
    if fin is not None:
        print('closing the file.')
        fin.close()
\end{pycode}
\afterverb

As you can see, the {\tt try} clause contains only the code that may raise an exception. 
\begin{itemize}
	\item If an exception occurs whilst opening the file, the program jumps to the {\tt except} clause and then executes the {\tt finally} clause. In this case, the output is: 

\begin{pyoutput}
TRY: open file.
EXCEPT: Something went wrong.
FINALLY: cleaning up.
\end{pyoutput}

	\item If no exceptions are raised, the program skips the {\tt except} clause, and jumps to the {\tt else} clause. Once the code in the {\tt else} clause has been executed, the program jumps to the {\tt finally} clause. In this case, the output is: 

\begin{pyoutput}
TRY: open file.
--> file open successfully.
ELSE: Do your thing if all went well.
--> line 1 of text file

--> line 2 of text file

--> last line of text file

FINALLY: cleaning up.
--> closing the file.
\end{pyoutput}

\end{itemize} 

\section{Writing modules}
\label{modules}

\index{module, writing}
\index{word count}

Any file that contains Python code can be imported as a module.
For example, suppose you have a file named {\tt wc.py} with the following
code:

\beforeverb
\begin{pycode}
def linecount(filename):
    count = 0
    for line in open(filename):
        count += 1
    return count

print(linecount('wc.py'))
\end{pycode}
\afterverb
%
If you run this program, it reads itself and prints the number
of lines in the file, which is 7.
You can also import it like this:

\beforeverb
\begin{pyinterpreter}
>>> import wc
7
\end{pyinterpreter}
\afterverb
%
Now you have a module object {\tt wc}:

\index{module object}
\index{object!module}

\beforeverb
\begin{pyinterpreter}
>>> print(wc)
<module 'wc' from 'wc.py'>
\end{pyinterpreter}
\afterverb
%
That provides a function called \verb"linecount":

\beforeverb
\begin{pyinterpreter}
>>> wc.linecount('wc.py')
7
\end{pyinterpreter}
\afterverb
%
So that's how you write modules in Python.

The only problem with this example is that when you import
the module it executes the test code at the bottom.  Normally
when you import a module, it defines new functions but it
doesn't execute them.

\index{import statement}
\index{statement!import}

Programs that will be imported as modules often
use the following idiom:

\beforeverb
\begin{pycode}
if __name__ == '__main__':
    print(linecount('wc.py'))
\end{pycode}
\afterverb
%
\verb"__name__" is a built-in variable that is set when the
program starts.  If the program is running as a script,
\verb"__name__" has the value \verb"__main__"; in that
case, the test code is executed.  Otherwise,
if the module is being imported, the test code is skipped.

\begin{exercise}
Type this example into a file named {\tt wc.py} and run
it as a script.  Then run the Python interpreter and
{\tt import wc}.  What is the value of \verb"__name__"
when the module is being imported?

Warning: If you import a module that has already been imported,
Python does nothing.  It does not re-read the file, even if it has
changed.

\index{module!reload}
\index{reload function}
\index{function!reload}

If you want to reload a module, you can use the built-in function 
{\tt reload}, but it can be tricky, so the safest thing to do is
restart the interpreter and then import the module again.
\end{exercise}



\section{Debugging}

\index{debugging}
\index{whitespace}

When you are reading and writing files, you might run into problems
with whitespace.  These errors can be hard to debug because spaces,
tabs and newlines are normally invisible:

\beforeverb
\begin{pyinterpreter}
>>> s = '1 2\t 3\n 4'
>>> print(s)
1 2	 3
 4
\end{pyinterpreter}
\afterverb

\index{repr function}
\index{function!repr}
\index{string representation}

The built-in function {\tt repr} can help.  It takes any object as an
argument and returns a string representation of the object.  For
strings, it represents whitespace
characters with backslash sequences:

\beforeverb
\begin{pyinterpreter}
>>> print(repr(s))
'1 2\t 3\n 4'
\end{pyinterpreter}
\afterverb

This can be helpful for debugging.

One other problem you might run into is that different systems
use different characters to indicate the end of a line.  Some
systems use a newline, represented \verb"\n".  Others use
a return character, represented \verb"\r".  Some use both.
If you move files between different systems, these inconsistencies
might cause problems.

\index{end of line character}

For most systems, there are applications to convert from one
format to another.  You can find them (and read more about this
issue) at \url{wikipedia.org/wiki/Newline}.  Or, of course, you
could write one yourself.


\section{Glossary}
	
\begin{vocabulary}[persistent:] Pertaining to a program that runs indefinitely
and keeps at least some of its data in permanent storage.
\index{persistence}
\end{vocabulary}
	
\begin{vocabulary}[format operator:] An operator, {\tt \%}, that takes a format
string and a tuple and generates a string that includes
the elements of the tuple formatted as specified by the format string.
\index{format operator}
\index{operator!format}
\end{vocabulary}
	
\begin{vocabulary}[format string:] A string, used with the format operator, that
contains format sequences.  
\index{format string}
\end{vocabulary}
	
\begin{vocabulary}[format sequence:] A sequence of characters in a format string,
like {\tt \%d}, that specifies how a value should be formatted.
\index{format sequence}
\end{vocabulary}
	
\begin{vocabulary}[text file:] A sequence of characters stored in permanent
storage like a hard drive.
\index{text file}
\end{vocabulary}
	
\begin{vocabulary}[directory:] A named collection of files, also called a folder.
\index{directory}
\end{vocabulary}
	
\begin{vocabulary}[path:] A string that identifies a file.
\index{path}
\end{vocabulary}
	
\begin{vocabulary}[relative path:] A path that starts from the current directory.
\index{relative path}
\end{vocabulary}
	
\begin{vocabulary}[absolute path:] A path that starts from the topmost directory
in the file system.
\index{absolute path}
\end{vocabulary}
	
\begin{vocabulary}[catch:] To prevent an exception from terminating
a program using the {\tt try}
and {\tt except} statements.
\end{vocabulary}


\section{Exercises}
\label{sec:files-exercises}

\begin{exercise}
\label{urllib}

\index{urllib module}
\index{module!urllib}
\index{URL}

The {\tt urllib} module provides methods for manipulating URLs
and downloading information from the web.  The following example
downloads and prints a secret message from {\tt thinkpython.com}:

\beforeverb
\begin{pyexo}
import urllib

conn = urllib.urlopen('http://thinkpython.com/secret.html')
for line in conn.fp:
    print(line.strip())
\end{pyexo}
\afterverb

Run this code and follow the instructions you see there.

\index{secret exercise}
\index{exercise, secret}

\end{exercise}

\begin{exercise}
\label{ex:words-file}
Write a program that reads {\tt words.txt} and prints only the
words with more than 20 characters (not counting whitespace).

\index{whitespace}

\end{exercise}

\begin{exercise}
\label{ex:gadsby}
In 1939 Ernest Vincent Wright published a 50,000 word novel called
{\em Gadsby} that does not contain the letter ``e.''  Since ``e'' is
the most common letter in English, that's not easy to do.

In fact, it is difficult to construct a solitary thought without using
that most common symbol.  It is slow going at first, but with caution
and hours of training you can gradually gain facility.

All right, I'll stop now.

Write a function called \verb"has_no_e" that returns {\tt True} if
the given word doesn't have the letter ``e'' in it.

Modify your program from the previous section to print only the words
that have no ``e'' and compute the percentage of the words in the list
have no ``e.''

\index{lipogram}

\end{exercise}


\begin{exercise} 
\label{ex:avoids}
Write a function named {\tt avoids}
that takes a word and a string of forbidden letters, and
that returns {\tt True} if the word doesn't use any of the forbidden
letters.

Modify your program to prompt the user to enter a string
of forbidden letters and then print the number of words that
don't contain any of them.
Can you find a combination of 5 forbidden letters that
excludes the smallest number of words?
\end{exercise}



\begin{exercise}
\label{ex:uses-only}

Write a function named \verb"uses_only" that takes a word and a
string of letters, and that returns {\tt True} if the word contains
only letters in the list.  Can you make a sentence using only the
letters {\tt acefhlo}?  Other than ``Hoe alfalfa?''
\end{exercise}


\begin{exercise} 
\label{ex:uses-all}

Write a function named \verb"uses_all" that takes a word and a
string of required letters, and that returns {\tt True} if the word
uses all the required letters at least once.  How many words are there
that use all the vowels {\tt aeiou}?  How about {\tt aeiouy}?
\end{exercise}


\begin{exercise}
\label{ex:abecedarian}

Write a function called \verb"is_abecedarian" that returns
{\tt True} if the letters in a word appear in alphabetical order
(double letters are ok).  
How many abecedarian words are there?
\end{exercise}

\index{abecedarian}


\begin{exercise}
\label{ex:palindrome}
A palindrome is a word that reads the same
forward and backward, like ``rotator'' and ``noon.''
Write a boolean function named \verb"is_palindrome" that
takes a string as a parameter and returns {\tt True} if it is
a palindrome.

Modify your program from the previous section to print all
of the palindromes in the word list and then print the total
number of palindromes.
\end{exercise}

